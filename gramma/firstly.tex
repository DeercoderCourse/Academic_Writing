\documentclass{article}

\title{First, firstly or at first? ~\\ from English Grammar Today }

\begin{document}

\maketitle

First and firstly


First can be an adjective or an adverb and refers to the person or thing that comes before all others in order, time, amount, quality or importance:

What’s the name of the first person who walked on the moon? (adjective)

Beth always arrives first at meetings. (adverb)

We often use first, especially in writing, to show the order of the points we want to make. When we are making lists, we can use first or firstly. Firstly is more formal than first:

Dear Mr Yates

First(ly) I would like to thank you for your kind offer of a job …

Not: At first I would like to thank you …

First(ly) the sodium chloride is dissolved in the water and heated gently. Second(ly) a dye is added to the solution.

Not: At first, the sodium chloride …

At first
At first means ‘at the beginning’ or ‘in the beginning’ and we use it when we make contrasts:

At first when I went to England to study English, I was homesick, but in the end I cried when it was time to leave.

He called for help. No one heard him at first, but eventually two young girls came to help him.


\end{document}