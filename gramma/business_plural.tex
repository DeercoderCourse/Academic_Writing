\documentclass{article}

\title{Business and businesses}

\begin{document}



 

Q:Does business in the singular have a different meaning from business in the plural?


A:

Business can be a singular or plural count noun, and also a noncount noun.

As a count noun, businesses—with an –es ending—is the plural of business. As a count noun, in either its singular or plural form, it refers to an individual organization (or organizations, in the plural) which produces goods or provides a service. It is similar to "company," as in these sentences:

(a) His family has a real estate agency. It's a good business.

(b) Barbara and John are looking for a small business to buy.

(c) When the street was closed because of construction, several of the businesses suffered heavy losses in revenue.

As a noncount noun, business has a more abstract meaning; it refers to the idea of commerce as in these sentences:

(d) Business is bad these days.

(e) Betty majored in business.

(f) George Bush studied business at Harvard.

Many nouns can be both count and noncount, as in sentences (g) and (h):

(g) Marisa has had a lot of rewarding experiences teaching handicapped children.

(h) Marisa has had a lot of experience teaching handicapped children.

Sentence (g)—with experiences as a count noun—refers to individual experiences, the many separate experiences, that Marisa has had teaching handicapped children.

Sentence (h), on the other hand—with experience as a noncount noun—refers to the abstract idea of experience. When you use experience in this way, you are not thinking about each experience individually, but the total idea of experience.

The noun light is another example:

(i) One light in our kitchen ceiling keeps burning out; we're replacing all our kitchen lights with a new kind.

(j) Light is necessary for plants to live.

Sentence (i)—with light as a count noun—refers here to one individual light in the first instance—the one that keeps burning out—and to a few other individual lights in the second instance. These lights can be visualized and counted. Sentence (j), on the other hand, refers to light as an idea, an abstraction.

Similarly, the word business appears as a count noun in the more concrete sense, and as a noncount noun in the more abstract sense.

Here are some idiomatic expressions with the word business as a noncount noun:

(k) Now we're in business.

(l) Don't fool around with him—he means business.

(m) You should make it your business to find out what really happened.

(n) A lot of dot.com companies went out of business.

(o) I didn't ask him because it's none of my business.



\end{document}