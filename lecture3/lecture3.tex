\documentclass{article}
\usepackage{indentfirst}
\title{Academic Writing ~\\ Lecture III}
\author{Chang Liu}


\begin{document}

\maketitle

\section{Part 1: some correct ways - STYLE}

see P151/P150 for details

1. It's not good to use `\textbf{etc}' when there're less than 3 examples.

2. lots of $\rightarrow$ numerous (`\textbf{lots of}' is always not good)

3. Never use `\textbf{I think}', because it's too subjective, try to use \textbf{`assume', `conclude', `result shows', `it works' etc}

4. Don't use `\textbf{we}', it's also too subjective, try to use `\textbf{our research group/center}' instead.

5. Don't use `\textbf{quite}', it is also too subjective, try to use `\textbf{significantly}'

6. `\textbf{it is OK}' is also not so good, `\textbf{when we think about}' is also not so objective.

7. some rhetorical questions like `\textbf{So how do we increase production}'

8. Remember that too \textbf{informal} or \textbf{personal} should not work, just remember these tips.`

\section{Part 2: Practices}

1. Another thing to think about is the chance of crime getting worse.

$\rightarrow$ another \underline{fact} to \underline{consider} is the \underline{possibility} of crime \underline{increase}.

2. Regrettably these days lots of people don't have jobs.

$\rightarrow$ currently, many people are unemployed.

3. Sometimes soon they will find a vaccine for malaria.

$\rightarrow$ In the near future, a vaccine would be found.

4. A few years ago the price of property in Japan went down a lot.

$\rightarrow$ In the 1990s, the price of property in Japan decrease significantly.


\section{Part 3: Paragraph}

see \textbf{P78} for more details.

There should a clear structure for an article, which include \textbf{Introduction}, \textbf{Body} and \textbf{Conclusion}. And Introduction and conclusion should catch the reader's eyes, that's the most important part in the article. In the body, we give evidence, proof, examples to support our assumptions.

For the thesis, the begin should contain the three elements. And for \textbf{multiple paragraphs}, we should also has topic sentences, body and conclusion.  Even for a single paragraph, it's recommend to organize in similar ways.

We should have structures like \textbf{topic}, \textbf{example}, \textbf{reason}, \textbf{supporting point}, \textbf{conclusion}. Many similar sentences form a paragraph, and many paragraphs form an article. Sometimes the last supporting points(for example, above all), is also the conclusion sentence.

And try to remember some good practice, for example, ``German, for instance ....'', some long sentence with short sentence seperated them are very catchy, try to do it in this way.

The last advice is to use them in daily life and try to read in this way to find good practice, try to write our own paper in this way and keep rewriting and eliminating the mistakes and not-so-good practice in our thesis!

\end{document}