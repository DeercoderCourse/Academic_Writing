\documentclass{article}
\usepackage[utf8]{inputenc}
\usepackage[english]{babel}
 
\title{02.507 Academic Writing Homework 6}
%\setlength{\parindent}{4em}
%\setlength{\parskip}{1em}
%\renewcommand{\baselinestretch}{1.0}

\author{Chang Liu ~\\ chang\_liu@student.uml.edu} 

\begin{document}

\maketitle

\section{Problem}

Write a paragraph using cause-effect structure.

\section{Writing}

\large{

There are many reasons why more and more scientists are putting their efforts on the research of machine learning. Over the last few decades, because of the limited size of training data set, it's not plausible to get an accurate classifier to predict the labels for various images. However, the situation has been greatly changed recently due to the rapid development of crowd sourcing and human labelling work. With more and more labelled images, now researchers can take advantage of numerous images and generate precise classifiers for object detection in the images. This is the main reason why until recently the machine learning techniques have been greatly improved. On the other hand, because of the combination of applied math and practical machine learning techniques, scientists have proved that large amounts of training set will actually improve the accuracy if it is big enough, so more and more institutes adjust their budget on this project and put more money and resource on the data preparation and collection. At last, the increasing need of secure network in surveillance and artificial intelligence has raised more challenges for the bottleneck of current machine learning research. Based on all these reasons, it's not surprising that machine learning has made such tremendous breakthrough during these years and it's being widely used almost everywhere.

}



\end{document}