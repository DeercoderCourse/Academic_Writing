\documentclass{article}
\usepackage[utf8]{inputenc}
\usepackage[english]{babel}
 
\title{02.507 Academic Writing Homework 5}
%\setlength{\parindent}{4em}
%\setlength{\parskip}{1em}
%\renewcommand{\baselinestretch}{1.0}

\author{Chang Liu ~\\ chang\_liu@student.uml.edu} 

\begin{document}

\maketitle

\section{Problem}

Write a paragraph on the topic: `Is the exploration of space worthwhile?' Use the ideas from P108 and make your stance clear.

\section{Writing}

\large{

As far as concerned, more and more countries are paying much attention on the exploration of space. But, is this worthwhile? Among scientists it's widely agreed that space exploration is an essential way to collect information to understand universe. By developing the space technology and engineering, many useful discoveries(e.g satellite) have also been produced as well, thus promotes the applying of modern science. However, the input-output ratio is strongly doubted by many experts, since huge amounts of money are spent in this area with very little result. With this amount of money, much more results can be achieved in other areas related with actual needs on earth. And many people agreed that resources should be spent on the urgent needs on earth to improve our living standards and push forward the development of science on earth. At last, despite that some advocates may argue that the exploration will boost the healthy cooperation between nations, it's still clear that national space programmes are testing potential weapons in the meantime. In most cases, countries have to put more efforts in the competition in case that they may fall behind on the resource exploitation and weapon development, which leaves a big challenge on the world peace and breaks the trust between nations. Overall, in terms of the benefits with regard to the disadvantages it brings, it shows that space exploration is overrated and its effect should be re-evaluated by the actual outcome. 

}



\end{document}