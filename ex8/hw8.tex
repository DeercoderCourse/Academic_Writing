\documentclass{article}
\usepackage[utf8]{inputenc}
\usepackage[english]{babel}
 
\title{02.507 Academic Writing Homework 8}
%\setlength{\parindent}{4em}
%\setlength{\parskip}{1em}
%\renewcommand{\baselinestretch}{1.0}
\author{Chang Liu ~\\ chang\_liu@student.uml.edu} 

\begin{document}

\maketitle

\section{Problem}

Find an abstract, and analyze its writing style, list its good or bad practice and analysis them in details.

\section{Abstract}
We propose a deep convolutional neural network architecture codenamed Inception,
which was responsible for setting the new state of the art for classification
and detection in the ImageNet Large-Scale Visual Recognition Challenge 2014
(ILSVRC14). The main hallmark of this architecture is the improved utilization
of the computing resources inside the network. This was achieved by a carefully
crafted design that allows for increasing the depth and width of the network while
keeping the computational budget constant. To optimize quality, the architectural
decisions were based on the Hebbian principle and the intuition of multi-scale
processing. One particular incarnation used in our submission for ILSVRC14 is
called GoogLeNet, a 22 layers deep network, the quality of which is assessed in
the context of classification and detection.


\section{Analysis}


\large{

I think this abstract is well-written, precise and clear. By using an unstructured way, even though we cannot
see clearly the result and conclusion from this short abstract, we can still grasp the main point in their paper.
Firstly, they proposed a new terminology called ``Inception'', which highlights their contribution that achieves
the state-of-art classification and detection in ILSVRC14. Then the advantage of this architecture and the theory
behind it are given to illustrate the reason why it can improve the accuracy significantly. To make it more convincing,
in the last part, the mathematical principle and process are put forward to better explain it. At last, they proposed
an example called ``GoogLeNet'' to show the exact application using their techniques. Overall I think this abstract
has shown the essential part in their paper.

In terms of the structure, as illustrated above, the abstract is organized in unstructured way, because they're formed
using the transition instead of the exact components for a structured abstract like ``Objective'', ``Method'', ``Result''
and ``Conclusion''. I think they write this abstract using their internal relationship like cause and effect, result and
reason. This abstract focuses more on the result and emphasis on its theory and exact application, but ignoring the experiment
and result. I think they can add some short description at the last part to make it more complete.


}



\end{document}