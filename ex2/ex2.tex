\documentclass{article}
\usepackage{indentfirst}
\title{Academic Writing Homework II}
\author{Chang Liu ~\\ chang\_liu@student.uml.edu}


\begin{document}

\maketitle

\section{Description}

Try to find the best paragraph from a paper in your field by your professor's recommendation, analyse how it's written and where is the high-lighted point, for example, you can extend from the point view of transition, style in that paragraph, describe why you like or dislike it.

\section{Reference} 
``Our results seem to yield a solid evidence that approximating the expected optimal sparse structure
by readily available dense building blocks is a viable method for improving neural networks for
computer vision. The main advantage of this method is a significant quality gain at a modest increase
of computational requirements compared to shallower and less wide networks. Also note that
our detection work was competitive despite of neither utilizing context nor performing bounding box
regression and this fact provides further evidence of the strength of the Inception architecture. Although
it is expected that similar quality of result can be achieved by much more expensive networks
of similar depth and width, our approach yields solid evidence that moving to sparser architectures
is feasible and useful idea in general. This suggest promising future work towards creating sparser
and more refined structures in automated ways on the basis of [2].''


\section{Comment}
The selected paragraph is the conclusion part from the paper named ``Going deeper with convolutions''. I think this part is well-written because of its concise expression and good transition. The detailed analysis is as follows:

1. `seem to yield a solid evidence' is an impressive expression. By avoiding the absolute and less-convincing statement, it delivers a hedging attitude about their findings, the phrase more professional than `prove' or `show' in writing a scientific paper or thesis.

2. The first long sentence has a very complex structure, containing different preposition phrases. `approximating ... structure by ... is a viable method for ...' is more impressive than `It's a viable method that ...'. The combination of these phrases quickly gained my attention and yield a significant emphasis on the topic for this paragraph.

3. `The main advantage is a significant quality gain at ... compared to ...' also catches my eye at first glance. The word `gain' is more precise and academic in the writing, and `compared to' gives us the comparison of their experiment result with others, the combination avoid the use of multiple short sentences, thus giving us a more precise practice.

4. `This fact provides further evidence' is also better than `This fact also shows that ...'.

5. `Although it's expected that similar .... can be achieved', this sentence has a good transition using `although' to emphasis their innovation and contribution in the experiment, and `achieve' is also more formal and academic in stating their findings.

6. There is a topic sentence in this paragraph, which is the first sentence, showing their method of `sparse structure' in computer vision, later in this paragraph, the author gives us a summary of the advantage, contribution and innovation to support this topic. At last a future work is proposed to show their extension and research direction. Overall this paragraph is clearly structured and precisely expressed.

\section{Summary}
From this paragraph, I learned how to use academic words to express our result, conclusion and propose. Another skill that I mastered is the combination of short and long sentence with proper preposition. The clear structure and organization in this paragraph also yield a new conception about how to emphasis the topic and to provide support for it. It's a good exercise and review of the main point in previous academic writing class.

\end{document}